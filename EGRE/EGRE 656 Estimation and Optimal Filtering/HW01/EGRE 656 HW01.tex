\documentclass{article}
\usepackage[utf8]{inputenc}
\usepackage[english]{babel}
\usepackage{amssymb,amsmath,amsthm}

\title{EGRE-656 Homework 1}
\author{John Hiles}
\date\today

\begin{document}
\maketitle %This command prints the title based on information entered above


\section*{Problem 1}
Let $A = \begin{bmatrix}
1 & 4 & -3 \\
6 & 3 & 0
\end{bmatrix}$, and $B=\begin{bmatrix}
3 & 2 & 1 \\
-2 & -6 & 5
\end{bmatrix}$.
\subsection*{Part A}
\begin{align*}
A + B = \begin{bmatrix}
1 & 4 & -3 \\
6 & 3 & 0
\end{bmatrix} + \begin{bmatrix}
3 & 2 & 1 \\
-2 & -6 & 5
\end{bmatrix} = \begin{bmatrix}
1+3 & 4+2 & -3+1 \\
6-2 & 3-6 & 0+5
\end{bmatrix} =
\begin{bmatrix}
4 & 6 & -2 \\
4 & -3 & 5
\end{bmatrix}
\end{align*}
\subsection*{Part B}
\begin{align*}
4A - 3B = \begin{bmatrix}
(4)1 & (4)4 & (4)-3 \\
(4)6 & (4)3 & (4)0
\end{bmatrix} + \begin{bmatrix}
(3)3 & (3)2 & (3)1 \\
-2(3) & -6(3) & (3)5
\end{bmatrix} = \begin{bmatrix}
4-9 & 16-6 & -12-3 \\
24+6 & 12+18 & 0+15
\end{bmatrix}
\end{align*}
\begin{align*}
\boxed{4A - 3B =
\begin{bmatrix}
-5 & 10 & -15 \\
30 & 30 & 15
\end{bmatrix} }
\end{align*}

\section*{Problem 2}
Compute the product of the two matricies below:
\begin{align*}
A = \begin{bmatrix}
2 & 5 \\
-1 & 3 \\
2 & -2
\end{bmatrix} && \boxed{ B = \begin{bmatrix}
1 & 5 \\
2 & 0
\end{bmatrix}}
\end{align*}

\subsection*{Part A}
Matrix multiplication is given by:
\begin{align*}
(AB)_{ij} = \sum_{k=1}^{2} a_{ik} b_{kj}
\end{align*}
Where $ij$ denotes the element on the $ith$ row and the $jth$ column. Note here that $i$ ranges from $1$ to $3$ and $j$ ranges from $1$ to $2$ due to the rows of $A$ and the columns of $B$ respectively.
Thus our product is written as:
\begin{align*}
AB = \begin{bmatrix}
\sum_{k=1}^{2} a_{1k} b_{k1} & \sum_{k=1}^{2} a_{1k} b_{k2} \\
\sum_{k=1}^{2} a_{2k} b_{k1} & \sum_{k=1}^{2} a_{2k} b_{k2} \\
\sum_{k=1}^{2} a_{3k} b_{k1} & \sum_{k=1}^{2} a_{3k} b_{k2}
\end{bmatrix}
\end{align*}
This looks horrible. And that's because it is. This is why we have computers.
\begin{align*}
AB = \begin{bmatrix}
a_{11} b_{11} + a_{12} b_{21} & a_{11} b_{12} + a_{12} b_{22}\\
a_{21} b_{11} + a_{22} b_{21} & a_{21} b_{12} + a_{22} b_{22}\\
a_{31} b_{11} + a_{32} b_{21} & a_{31} b_{12} + a_{32} b_{22}
\end{bmatrix} = \begin{bmatrix}
2 (1) + 5 (5) & 2 (5) + 5 (0)\\
(-1) (1) + 3 (2) & (-1) (5) + 3 (0)\\
2 (1) + (-2) (2) & 2 (5) + (-2) (0)
\end{bmatrix}
\end{align*}
Our final product is given by:
\begin{align*}
\boxed{AB = \begin{bmatrix}
12 & 10\\
5 & -5 \\
-2 & 10
\end{bmatrix}}
\end{align*}

\section*{Problem 3}
If it exists, find the inverse of $A$, by using $|A|$, and its adjugate.
\begin{align*}
A = \begin{bmatrix}
1 & 0 & 1\\
1 & 1 & 1\\
2 & -1 & 1
\end{bmatrix}
\end{align*}
\subsection*{Part A}
Note here that the inverse of $A$ is:
\begin{align*}
A^{-1} = \frac{1}{|A|} C^T \text{ where C is the cofactor matrix of A} 
\end{align*}
Its important to note at this stage that if $|A| = 0$, this fraction is undefined and thus there is no inverse. For brevity, I will proceed without giving the procedure for computing a determinate or the composition of the cofactor matrix. So long as $|A|$ exists, we are guaranteed the existence of an inverse since a cofactor matrix is always defined for a square matrix.
\begin{align*}
|A| = -1 && C = \begin{bmatrix}
2 & 1 & -3\\
-1 & -1 & 1\\
-1 & 0 & 1
\end{bmatrix}
\end{align*}
Thus, our inverse is then given by
\begin{align*}
\boxed{A^{-1} = \frac{1}{|A|} C^T = (-1) \begin{bmatrix}
2 & 1 & -3\\
-1 & -1 & 1\\
-1 & 0 & 1
\end{bmatrix}^T = \begin{bmatrix}
-2 & 1 & 1\\
-1 & 1 & 0\\
3 & -1 & -1
\end{bmatrix}}
\end{align*}



\end{document} 

\documentclass[conference]{IEEEtran}
\IEEEoverridecommandlockouts
% The preceding line is only needed to identify funding in the first footnote. If that is unneeded, please comment it out.
\usepackage{cite}
\usepackage{amsmath,amssymb,amsfonts}
\usepackage{algorithmic}
\usepackage{graphicx}
\usepackage{textcomp}
\usepackage{xcolor}
\def\BibTeX{{\rm B\kern-.05em{\sc i\kern-.025em b}\kern-.08em
    T\kern-.1667em\lower.7ex\hbox{E}\kern-.125emX}}
\begin{document}

\title{Conference Paper Title*\\
{\footnotesize \textsuperscript{*}Note: Sub-titles are not captured in Xplore and
should not be used}
\thanks{Identify applicable funding agency here. If none, delete this.}
}

\author{\IEEEauthorblockN{John Hiles}
\IEEEauthorblockA{Virginia Commonwealth Univ.\\
Richmond, VA,  USA  \\
hilesj@vcu.edu}
\and
\IEEEauthorblockN{Liang Xu}
\IEEEauthorblockA{WORKPLACE\\
ADDRESS  \\
EMAIL}
\and
\IEEEauthorblockN{Ruixin Niu}
\IEEEauthorblockA{Virginia Commonwealth Univ.\\
Richmond, VA, USA  \\
rniu@vcu.edu}
\and
\IEEEauthorblockN{Erik P. Blasch}
\IEEEauthorblockA{AF Office of Scientific Research\\
	Arlington, VA, USA  \\
	erik.blasch.1@us.af.mil}
}


\maketitle

\begin{abstract}
This document is a model and instructions for \LaTeX.
This and the IEEEtran.cls file define the components of your paper [title, text, heads, etc.]. *CRITICAL: Do Not Use Symbols, Special Characters, Footnotes, 
or Math in Paper Title or Abstract.
\end{abstract}

\begin{IEEEkeywords}
component, formatting, style, styling, insert
\end{IEEEkeywords}

\section{Introduction}
This document is a model and instructions for \LaTeX.
Please observe the conference page limits. 

\section{Maneuvering Target Problem}
The maneuvering target problem is an area of great interest for modern target tracking. It is characterized by estimating the motion of an object in which its dynamics are unknown to the observer, but the set of possible ways the target can move are known or partially known. In particular, the object is capable of "maneuvers" typically represented by a change in motion model.

This process is best can be represented via a hidden markov model. <HIDDEN MARKOV MODELS>

The standard approach to estimating the trajectories of maneuvering targets is the multipule model (MM) approach. In a MM algorithm the target is assumed to move according to one of $M$ motion models at any given time step $k$. A MM algorithm runs some number of filters in parallel, while a non-MM based approach tends to make a decision to run one of many potential filters. 

The current state of the art MM algorithm is generally considered to be the IMM or Interacting Multipule Model Filter(IMM). The IMM runs $M$ filters in parallel, and after the posterior state of each of the M filters is calculated, the $M$ posterior states are blended together according to $EQUATION$. The key feature here is the Transition Probability Matrix which encodes all the information about the system's probabilities to switch or not switch from one model to another.

The IMM is renown for its computational efficiency. It offers performance comparable to the first order Generalized Pseudo Bayesian filter while offering performance comparable to the second order Generalized Pseudo Bayesian Filter($ON^2$). $FIGURE or EQUATIONS SHOWING ALL 3 COMPLEXITY$.

One of the core difficulties with maneuvering target tracking is its difficulty in detecting model changes. All manuvering target algorithms see a high spike in estimation error during a model switch. After error spikes, these algorithms tend to plateau off to a new equilibrium of error until the next model transition happens again, causing another spike in the error.

Machine Learning represents a promising new avenue for improvement in these methods and much attention has been given to finding novel ways to incorporate machine learning$FUSION 7920$. Neural Networks have been employed to train on simulated datasets to allow networks to learn state dynamics$Paper num 2205$ as well as noise statistics$EKF Net$. But one important property of neural networks which hasn't been leveraged in current tracking research is the input agnostic nature of neural networks.

\subsection{LSTM Neural Networks}

The network architecture chosen for this simulation is the Long Short Term Memory or LSTM. LSTM neural networks are a type of Recurrent Neural Network characterized by the ability to keep and "discard" internal information pertaining to the system state while assimilating new data. This property is of particular relevance to maneuvering target tracking as it means that the LSTM is particularly well suited to Hidden Markov Models, which the maneuvering target problem is a subfield of.



\subsubsection{Prediction Step}

\subsubsection{Update Step}

\subsection{Neural Network based Classification}

\section{Learning Enhanced IMM}


\subsection{Abbreviations and Acronyms}\label{AA}


\section{Dataset Generation}

\subsection{Dataset Generation}

\section{Results}


\subsection{Figures and Tables}

\section*{Acknowledgment}


\section*{References}


\end{document}

% This is samplepaper.tex, a sample chapter demonstrating the
% LLNCS macro package for Springer Computer Science proceedings;
% Version 2.21 of 2022/01/12
%
\documentclass[runningheads]{llncs}
%
\usepackage[T1]{fontenc}
% T1 fonts will be used to generate the final print and online PDFs,
% so please use T1 fonts in your manuscript whenever possible.
% Other font encondings may result in incorrect characters.
%
\usepackage{graphicx}
% Used for displaying a sample figure. If possible, figure files should
% be included in EPS format.
%
% If you use the hyperref package, please uncomment the following two lines
% to display URLs in blue roman font according to Springer's eBook style:
%\usepackage{color}
%\renewcommand\UrlFont{\color{blue}\rmfamily}
%\urlstyle{rm}
%
\begin{document}
%
\title{Transformer Based Context-Aware Maneuvering Ground Vehicle Trackers}
%
%\titlerunning{Abbreviated paper title}
% If the paper title is too long for the running head, you can set
% an abbreviated paper title here
%
\author{First Author\inst{1}\orcidID{0000-1111-2222-3333} \and
Second Author\inst{2,3}\orcidID{1111-2222-3333-4444} \and
Third Author\inst{3}\orcidID{2222--3333-4444-5555}}
%
\authorrunning{F. Author et al.}
% First names are abbreviated in the running head.
% If there are more than two authors, 'et al.' is used.
%
\institute{Princeton University, Princeton NJ 08544, USA \and
Springer Heidelberg, Tiergartenstr. 17, 69121 Heidelberg, Germany
\email{lncs@springer.com}\\
\url{http://www.springer.com/gp/computer-science/lncs} \and
ABC Institute, Rupert-Karls-University Heidelberg, Heidelberg, Germany\\
\email{\{abc,lncs\}@uni-heidelberg.de}}
%
\maketitle              % typeset the header of the contribution
%
\begin{abstract}
Maneuvering target tracking is a challenging problem from multiple perspectives. The
defacto standard algorithm for maneuvering target tracking is the Interacting Multiple Model
Filter (IMM). The IMM is a multi model filtering algorithm, which assumes that the target at any given time step, with a finite
probability switching among a finite number of motion models.
Here we analyze the application of Neural Networks to assimilates both measurement
data which the typical IMM can parse, and qualitative data or “contextual” information which
does not have a straightforward mathematical relationship which can be represented in a
Markov Transition model. The simulation we present involves tracking a “vehicle” in the
presence of bidirectional and unidirectional roads. In addition to knowledge of current speed,
the proposed tracker has qualitative knowledge of the current position: whether or not the
location is at or near an intersection, and whether or not the current street is unidirectional or
not. These qualitative features provide information which can be used to deduce the likelihood
of a model switch or eliminate the possibility of model switches which violate traffic laws.
In this paper we compare the performance of pre trained networks which involve offline
pre training via Mean Square Error of estimation with algorithms which are entirely online
which don’t rely on knowledge of the ground truth. The performance of these neural network
based trackers is quantified via Mean Square Error plots. Special attention is given to analyzing
the rate at which the tracker “detects” a model switch, a key drawback of multiple model based
approaches. The tendency of the networks to overestimate or underestimate the probability of
the target following a certain model is also measured.
Assimilation of qualitative data offers a promising reduction in the spiking estimation
error which accompanies a model switch as well as an increase in confidence in the correct
model.

\keywords{First keyword  \and Second keyword \and Another keyword.}
\end{abstract}
%
%
%
\section{Introduction}
\subsection{Maneuvering Target Problem}

BACKGROUND ON MANEUVERING TARGET TRACKING

MULTIPLE MODEL APPROACH

DRAWBACK BEING SPIKE IN ERROR DURING MODEL SWITCH, CAN POTENTIALLY LOSE TARGET IN CHALLENGING CIRCUMSTANCES


\subsubsection{\ackname} A bold run-in heading in small font size at the end of the paper is
used for general acknowledgments, for example: This study was funded
by X (grant number Y).

\subsubsection{\discintname}

\subsection{Transformers and Data driven trackers}

NEURAL NETWORKS HAVE SEEN WIDESPREAD USE ACROSS NUMEROUS FIELDS.

PROBLEMS SUITABLE FOR MACHINE LEARNING TEND TO BE ONES WITH LARGE AMOUNTS OF DATA AVAILABLE

RNNs AND LSTMs HAVE BEEN APPLIED TO TRACKING FOR MANY YEARS, MORE RECENTLY TRANSFORMER BASED NETWORKS HAVE SEEN SUCCESS AT TRACKING




%
% ---- Bibliography ----
%
% BibTeX users should specify bibliography style 'splncs04'.
% References will then be sorted and formatted in the correct style.
%
% \bibliographystyle{splncs04}
% \bibliography{mybibliography}
%
\begin{thebibliography}{8}
\bibitem{ref_article1}
Author, F.: Article title. Journal \textbf{2}(5), 99--110 (2016)

\bibitem{ref_lncs1}
Author, F., Author, S.: Title of a proceedings paper. In: Editor,
F., Editor, S. (eds.) CONFERENCE 2016, LNCS, vol. 9999, pp. 1--13.
Springer, Heidelberg (2016). \doi{10.10007/1234567890}

\bibitem{ref_book1}
Author, F., Author, S., Author, T.: Book title. 2nd edn. Publisher,
Location (1999)

\bibitem{ref_proc1}
Author, A.-B.: Contribution title. In: 9th International Proceedings
on Proceedings, pp. 1--2. Publisher, Location (2010)

\bibitem{ref_url1}
LNCS Homepage, \url{http://www.springer.com/lncs}, last accessed 2023/10/25
\end{thebibliography}
\end{document}

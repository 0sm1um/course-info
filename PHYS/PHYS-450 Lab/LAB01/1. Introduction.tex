\section{Introduction}
\subsection{The Physics}
	Galium Nitride (GaN) is a commonly used semiconductor which boasts a wide band gap and mechanically strong properties. It has a lower resistivity and higher thermal conductivity, which makes it a prime candidate to one day replace Silicon based semiconductors for power conversion, RF, and analog electronics. As such, a strong fundamental understanding of how to measure the number of charge carriers on a GaN crystal is a crucial component for evaluating the quality of a manufactured GaN crystal. If GaN is to be one day considered for mass produced consumer electronics, a simple and scalable method of measuring quality is needed, and in this paper I outline one such method.
	
	The Van der Pauw method is a technique proposed in the late 1950s specifically to measure the resistivity of a semiconductor, determine the number of free charge carriers, and determine the type of semiconductor (p-type or n-type) it is. It is a multi part process, the second part involving measurement of the semiconductor in a magnetic field. However this paper is only concerned with the first part of the measurements with no magnetic field present. The measurements in the presence of a magnetic field will be covered at a later date. The focus of this paper will be on calculating the resistivity, conductivity, and free charge carriers of a semiconductor.
	
	The method involves placing four electrical contacts on a planar semiconductor. These contacts should be oriented in a square as close to the perimeter of the semiconductor as possible. The ideal shape of the semiconductor would be a perfect square of negligible thickness. However, the method has proven to be reasonably effective even with suboptimal samples. The four electrical contacts then should be labeled 1 to 4 as pictured in figure 1.

\begin{figure}[hbt!]
\centering
\includegraphics[width=0.9\linewidth]{4PointConfig.png}
\caption{Four Point Configuration}\label{fig:ligmaballs}
\end{figure}

With the following labels an external voltage will be applied across terminal 1 and 2, then current is measured from contact 1 to contact 2, then voltage drop is measured from contacts 3 to 4. Via Ohm’s Law, the following relationship can be established:

\begin{equation}
	R_{12,34}=\dfrac{V_{34}}{I_{12}} 
\end{equation}

	Note, that the reciprocal measurement can be repeated, and the result aggregated together for more certain measurements of resistance.

\begin{equation}
	R_{34,12}=\dfrac{V_{12}}{I_{34}} 
\end{equation}

	Furthermore, reversed polarity measurements can be taken to increase the measurement sample size by a factor of two once again. Meaning that there are a total of four measurements across the sample “vertically”, and four for resistance measurements across the sample “horizontally”. Totaling eight unique measurements for the maximum precision without repeat measurements.
This combination of simplicity and relative precision makes the Van der Pauw method suitable for industrial automation, as I hope to demonstrate.

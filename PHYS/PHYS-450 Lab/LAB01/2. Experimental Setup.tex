\section{Experimental Setup}
Samples must be prepared by soldering electrical contacts on to roughly square crystals. It would be sensible to solder copper wires to each of these contact points, as it is far easier to manipulate multicolored wires as opposed to unmarked electrical contacts for the voltage and current measurements. The measurements can be set up with a pair of multimeters and a breadboard. Set the breadboard such that the current and voltage contacts can be substituted without breaking the rest of the circuit for quick measurements. The voltmeter connected in parallel with the sample, and the ammeter connected in series, as pictured in figure 2. In an industrial setting, it isn’t difficult to imagine how this could be fully automated.

\begin{figure}[hbt!]
\centering
\includegraphics[width=0.9\linewidth]{experimentalsetup.png}
\caption{Experimental Setup}\label{fig:ligma}
\end{figure}

Recording these measurements will yield two sets of voltage and current pairs, which can be divided to yield four resistance measurements.



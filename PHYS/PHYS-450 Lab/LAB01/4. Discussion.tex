\section{Discussion}
	The results of our Van der Pauw measurement yield the number of charge carriers as $6.9 \times 10^{18} \pm 6.2 \times 10^{14}$ $\dfrac{cm^2 s}{V}$, which is comfortably within the range of values specified by the manufacturer: $1 \times 10^{18} < n < 1 \times 10^{19}$. We found resistivity to be $6.2 \times 10^{-3} \pm 1.2 \times 10^{-6}$ $\Omega cm$, which is well below than the manufacturer’s specification of $\rho < 5 \times 10^{-2}$  $\Omega cm$. The value of resistivity is suspiciously low, but it is still within the realms of acceptability for a GaN crystal semiconductor.

	Through the Van der Pauw method, we can measure to a high degree of accuracy the resistivity $\rho$, conductivity $\sigma$, and number of charge carriers of Galium Nitride $n$. Now, we went into this experiment with a well defined method, and a specimen with known parameters. But the fact an individual can measure these properties with two multimeters of intermediate quality, a breadboard which cost less than a dollar, wires, and a household battery—equipment who’s sum total cost is less than a hundred dollars—is indicative of a highly cost efficient and reliable method of evaluating the quality of semiconductors.

	A potential source of error in our lab enviornment, was the questionable electrical contact both between the measurement probes and the soldered on electrical contacts.

	In an industrial environment, experimental error could be greatly reduced by automation. In a semiconductor production line, a statistical sample of wafers could be drawn from main production to serve as quality check crystals. These crystals could be cut down to shape, and electrical contacts soldered to the corners either by machine or by trained line workers. Then, a device or series of devices could take these electrical readings in an entirely automated process. Through automation, human error can be removed from the method and speed can controlled. The precision of the measurement method will then be limited by the sensitivity of the voltage and current measurement tools, which are cheap in comparison to the costs of the samples being evaluated themselves.
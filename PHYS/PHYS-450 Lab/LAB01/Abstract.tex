\begin{abstract}
	The Van der Paw method is an experimental method of accurately measuring the resistivity of a semiconductor of variable shape, so long as the sample is planar and solid. The method involves measuring the electric potential, as well as the electric current across four ohmic contacts. The measurement can be repeated additional times per sample, plus an additional four times with the opposite polarity for an aggregated 8 measurements to average together to use to derive the semiconductor’s resistivity, conductivity, and number of charge carriers.
\end{abstract}

\section{Results}
\subsection{Data Analysis}
	Our plot of wavelength versus Intensity can be easily transformed into a plot of Photon Energy versus Intensity via the De Broglie relationship (eq 1.1). The plot confirms our earlier assertion that our sample possesses the expected Exciton band at 3.4eV and a Red Defect Broad Band at 1.89eV. 
	
%PL CURVE

	Furthermore, the raw measurements of intensity at the Exciton and Broad Band peaks are displayed below.

%RAW INTENSITY MEASUREMENT

	As can be seen in the plots of our measurements, the error of these measurements are staggeringly low. The plot of intensity versus time for our measurements shows very little noise in our measurements. 

	The percent error for our measurements is at worst $2.5$ to at best $0.04$, and the error increases as the excitation decreases with what resembles an inverse relationship between error and excitation energy. This is probably due to the fact that the monochromatic itself is approaching the point its resolution is insufficient at lower excitation intensities. This would probably be a factor for an experimenter to consider for even lower intensities (an order of magnitude or two lower than done in this experiment), but for our purposes these measurements are more than accurate enough.
	
%ERROR PLOT	
	
	Ultimately, plotting the excitation energy versus PL intensity reveals that both the Exciton and Broad bands follow a linear relationship between excitation energy and the intensity of PL emissions. One thing to note is how the Broad Band measurements are consistently higher intensity versus the Exciton Band for the same excitation intensity.

%LOGLOG PLOT
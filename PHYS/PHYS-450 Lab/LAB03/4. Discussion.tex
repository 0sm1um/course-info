\section{Discussion}
	The root cause of PL is that photons contact the semiconductor, and create an electron-hole pair. This pair then due to the thermal energy of the sample is knocked free of the crystal lattice structure, and the electromagnetic causes the to contact each other where they annihilate, releasing photons. Not all of the excitation energy is actually transferred into the holes of the semiconductor, and some of the energy is transferred into the sample itself as thermal energy. Thus, the measured intensity of exciton annihilation will be less than the excitation energy.

	The intensity is higher at the defect band than the exciton band, which implies ,more excitons at the given energy level are being created/annihilated at the defect band than the exciton band. This suggests that there is a higher concentration of whatever defect is causing the Red PL band than there is 

	A possible cause for the fact that our observations deviate from prediction is the fact the manufacturer specified that during manufacturing, acceptors were unintentionally introduced to the sample. The donors could be getting excited as a normal GaN sample would, but the energy level falls to a lower ground state than they would otherwise due to the acceptors.
Another anomaly is the fact that both bands exhibit the same level of linear dependance, I suspect there is another variable at play which is systemically skewing the results. Due to the fact the broad band doesn’t exhibit the nonlinear increase in PL intensity with excitation intensity, it is reasonable to deduce that the ambient temperature of the room was too high (or the excitation energy was too low) for us to observe the saturation of the holes.

	Further experimentation is required to determine the cause of the increased intensity of the broad defect band. I suspect two variables which were not controlled could be a cause. First, the ambient light of the room could be having an effect on the monochromator’s intensity reading. We used a large curtain to enclose the room which the experimental setup resided in, but it is possible reflection from gaps in the curtain or light passing through the curtain altogether could have added to the intensity measurements along the defect band.
A better solution would be to enclose the experiment table underneath a tent like structure in which absolutely no light from the outside may enter. Furthermore, electric tape can be placed over indicator LEDs to further minimize interference from other light sources in the experimental setup.

	We also should measure PL intensity of multiple samples, some from the same manufacturer and others from different manufacturers. If samples from the same manufacturer show the same behavior, then we know our measurements are consistent. If samples from other manufacturers deviate from our current results, then that confirms to us that the culprit of the higher intensity broad bands is probably related to the manufacturing process of the sample. If the other samples show the same results, then the cause may be due to a flaw in our experimental setup, OR it serves as evidence we are observing behavior which is not predicted by theory.

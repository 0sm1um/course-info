%This is my super simple Real Analysis Homework template

\documentclass{article}
\usepackage[utf8]{inputenc}
\usepackage[english]{babel}
\usepackage{amssymb,amsmath,amsthm}
\usepackage{geometry}
\geometry{legalpaper, margin=1.5in}

\title{STAT-514 Homework 2}
\author{John Hiles}
\date\today
%This information doesn't actually show up on your document unless you use the maketitle command below

\begin{document}
\maketitle %This command prints the title based on information entered above

%Section and subsection automatically number unless you put the asterisk next to them.
\section*{Problem 1}
Let $X_1,...,X_5$ be a random sample of size 5 from a normal population with mean $0 $and variance $1$ and let $\bar{X}=(1/5)\sum_{i=1}^{5}X_i^2$. Let $X_6$ be another independent observation from the same population. Find the distributions of
\begin{enumerate}
\item[A.] $W=\sum_{i=1}^5 X_i^2$
\item[B.] $U=\sum_{i=1}^5 (X_i-X_{\bar{X}})^2$
\item[C.] $T_1 = \sqrt{5} \frac{X_6}{\sqrt{W}}$
\item[D.] $T_2 = 2\frac{(5\bar{X}^2+(X_6)^2)}{U}$

\end{enumerate}
\subsection*{Part A}
$W$ simply represents the non averaged sum of the squares of the sample random variables.
\begin{align*}
W=\sum_{i=1}^5 X_i^2 = X_1^2+X_2^2+X_3^2+X_4^2+X_5^2
\end{align*}.
Here lets note that these are standard normal distributions. As such, the sum of their squares (i.e. $W$) will be a chi squared random variable with $5$ degrees of freedom.
\begin{align*}
W = \mathcal{X}_5^2
\end{align*}
\subsection*{Part B}
$U$ here is simply the sample Variance except missing the scaling $\frac{1}{n-1}$ factor.
\begin{align*}
S = \frac{1}{4} U && U = (4)S^2
\end{align*}
The sample variance for a sample of standard normal of sample size $5$ is:
\begin{align*}
S^2 \sim \frac{1}{4} \mathcal{X}_{4}^{2}
\end{align*}
As such, $U$ is then:
\begin{align*}
\boxed{ U = \mathcal{X}_{4}^{2} }
\end{align*}
\subsection*{Part C}
Finding $T_1$ should just be a case of simple substitution.
\begin{align*}
T_1 = \sqrt{5} \frac{X_6}{\sqrt{W}}
\end{align*}
Substituting in W from part A:
\begin{align*}
T_1 = \sqrt{5} \frac{X_6}{\sqrt{\mathcal{X}_5^2}}
\end{align*}
Note here that we can rewrite this fraction to move that factor of $\sqrt{5}$ from the numerator to the denominator.
\begin{align*}
T_1 = \frac{X_6}{\sqrt{\tfrac{\mathcal{X}_5^2}{5}}} 
\end{align*}
Here we should note that this is the exact form of Student's $T$ distribution. Thus $T_1$ is a T distribution with $5$ degrees of freedom.
\begin{align*}
T_1 \sim t_{5}
\end{align*}
\subsection*{Part D}
Before proceeding, lets specify the distribution of $\bar{X}$.
\begin{align*}
\bar{X} \sim \mathcal{N}(0,\frac{1}{5})
\end{align*}
This result was dealt with extensively in part 1 of the last homework. With this in mind lets look at $T_2$
\begin{align*}
T_2 = 2\frac{(5\bar{X}^2+(X_6)^2)}{U}
\end{align*}
Here we have a sum of two squares of normal distributions. The former being nonstandard. But luckily that factor of 5 will help us out:
\begin{align*}
T_2 = 2\frac{(5\bar{X}^2+(X_6)^2)}{U} = 2\frac{((\sqrt{5}\bar{X})^2+X_6^2)}{U}
\end{align*}
The product of a scalar and a normal random variable is another gaussian. The mean scales by a factor of the scalar and the variance scales by the square of that scalar. As such $\sqrt{5}\bar{X} \sim \mathcal{0,1}$. Therefore the square of $\sqrt{5}\bar{X}$ is a Chi-Squared random variable.
\begin{align*}
T_2 = 2\frac{(\mathcal{X}^2_5+(X_6)^2)}{U}
\end{align*}
We also know that $X_6$ is standard normal so it too is a Chi-Squared random variable.
\begin{align*}
T_2 = 2\frac{(\mathcal{X}^2_5+\mathcal{X}^2_5)}{U}
\end{align*}
These Chi-Squared can be combined:
\begin{align*}
T_2 = T_2 = \frac{4(\mathcal{X}^2_5)}{U}
\end{align*}
Now to substitute in $U$ from part C.
\begin{align*}
\boxed { T_2 = \frac{4\mathcal{X}^2_5}{\mathcal{X}^2_4} }
\end{align*}
Here we can rearrange the numerator like in part C to pull that factor of 4 into the denominator.
\begin{align*}
\boxed { T_2 = \frac{\mathcal{X}^2_5}{\tfrac{\mathcal{X}^2_4}{4}} }
\end{align*}

\clearpage

\section*{Problem 2}
Show that the median of an F-distribution with $(\nu, \nu)$ degrees of freedom is 1. Also,
show that $Q_1Q_3 = 1$, where Q1 and Q3 are the first and third quartiles, respectively, of the $F(\nu,\nu)$
distribution.

Hint: You will have to use the fact that the reciprocal of an F random variable is also an F random
variable.

\subsection*{Part A}
First lets find the median of the F distribution with $(\nu, \nu)$ degrees of freedom. The median is defined as the value such that $\int_{-\infty}^{m} f_X(x) dx = \int_{m}^{\infty} f_X(x) dx = \frac{1}{2}$. The CDF of the F distribution is the 

    
\clearpage
\section*{Problem 3}
Suppose $X_1,X_2,...,X_n$ be a random sample from an exponential distribution with mean $\theta$.
\begin{enumerate}
\item[A.] Derive the density for the smallest order statistic $X_{(1)}$. Identify the distribution.
\item[B.] Show that $X_{(1)}$ is independent of $X_{(n)}-X_{(1)}$.
\end{enumerate}
\subsection*{Part A}
Theorem 5.5.2 gives the pdf of the jth order statistic as:
\begin{align*}
f_{X_{(j)}}(x) = \frac{n!}{(j-1)!(n-j)!} f_X(x) [F_X(x)]^{j-1} [1-F_X(x)]^{n-j}
\end{align*}
In this case $j=1$, leaving us with:
\begin{align*}
f_{X_{(j)}}(x) = \frac{n!}{(n-1)!} f_X(x) [1-F_X(x)]^{n-1} = n f_X(x) [1-F_X(x)]^{n-1}
\end{align*}
Our pdf is the exponential distribution with mean 1. As such we know the parameter $\lambda=1$. Our pdf is then given by:\begin{align*}
f_X = e^{-x} && x \geq 0
\end{align*}
Our CDF is then:
\begin{align*}
F_X = 1-e^{-x} && x \geq 0
\end{align*}
Substituting these in:
\begin{align*}
f_{X_{(j)}}(x) = n e^{-x} [1-(1-e^{-x})]^{n-1} = n e^{-x} [2-e^{-x}]^{n-1}
\end{align*}
This can be rearranged further:
\begin{align*}
f_{X_{(j)}}(x) = n e^{-nx} [2e^{-x}-1]^{n-1}
\end{align*}
\subsection*{Part B}


\clearpage
\section*{Problem 4}
 Let $X_1,...,X_n$ be a random sample of size $n$ from a population with pdf
 \begin{align*}
 f(x) = \begin{cases} 
      \frac{1}{\theta} & 0<x\leq \theta \\
       0 & \text{ otherwise } 
   \end{cases}
\end{align*}
Let $X_{(1)}<...< X_{(n)}$ be the order statistics. Show that $\frac{X_{(1)}}{X_{(n)}}$ and $X_{(n)}$ are independent
random variables.
\subsection*{Part A}
Its important to note that this is a uniform distribution with start point $0$ and endpoint $\theta$.
The pdf was given, and the cdf is known to be:
\begin{align*}
F_X = \frac{x}{\theta} && 0 \leq x \leq \theta
\end{align*}
Thus the jth order statistics are:
\begin{align*}
f_{X_{(j)}}(x) = \frac{n!}{(j-1)!(n-j)!} f_X(x) [F_X(x)]^{j-1} [1-F_X(x)]^{n-j}
\end{align*}
Substituting in CDF and PDF:
\begin{align*}
f_{X_{(j)}}(x) = \frac{n!}{(j-1)!(n-j)!} \frac{1}{\theta} [\frac{x}{\theta}]^{j-1} [1-(\frac{x}{\theta})]^{n-j}
\end{align*}
Thus, the first order $j=1$ statistics are given by:
\begin{align*}
f_{X_{(1)}}(x) = n \frac{1}{\theta} [1-(\frac{x}{\theta})]^{n-1}
\end{align*}
And the $n$th order $j=n$ statistics are given by:
\begin{align*}
f_{X_{(n)}}(x) = n \frac{1}{\theta} [\frac{x}{\theta}]^{n-1}
\end{align*}
Now the fraction of $\frac{X_{(1)}}{X_{(n)}}$.
\begin{align*}
\frac{X_{(1)}}{X_{(n)}} = \frac{n \frac{1}{\theta} [1-(\frac{x}{\theta})]^{n-1}}{n \frac{1}{\theta} [\frac{x}{\theta}]^{n-1}}
\end{align*}
Simplifying further:
\begin{align*}
\frac{X_{(1)}}{X_{(n)}} = \frac{[1-(\frac{x}{\theta})]^{n-1}}{[\frac{x}{\theta}]^{n-1}}
\end{align*}



\clearpage
\section*{Problem 5}
Suppose $X_1,...,X_n$ be a random sample from a $Beta(2,1)$ distribution.
\begin{enumerate}
\item[A.] Derive the density for the smallest order statistic $X_{(1)}$
\item[B.] Suppose $n = 3$. Compute the probability that $X_{(1)}$ exceeds the median of the distribution.
\item[C.] Again, let $n = 3$. What is the covariance between $X_{(2)}$ and $X_{(3)}$.

\end{enumerate}

\subsection*{Part A}
All $X_{(n)}$ samples are sampled from the $Beta(2,1)$ distribution. The pdf and cdf of this distribution is given by:
\begin{align*}
f_X(x) = \frac{x^{\alpha-1}(1-x)^{\beta-1}}{B(\alpha,\beta)} = 2x & 0<x<1 && F_X(x) = B(2,1) = x^2
\end{align*}
Note here the CDF is the incomplete beta function. Theorem 5.5.2 gives the pdf of the jth order statistic as:
\begin{align*}
f_{X_{(j)}}(x) = \frac{n!}{(j-1)!(n-j)!} f_X(x) [F_X(x)]^{j-1} [1-F_X(x)]^{n-j}
\end{align*}
In this case $j=1$. Substituting in the CDF and PDF:
\begin{align*}
f_{X_{(j)}}(x) = \frac{n!}{(n-1)!} 2x [x^2]^{0} [1-x^2]^{n-1} = n 2x [1-x^2]^{n-1}
\end{align*}
Our density then is:
\begin{align*}
\boxed{ f_{X_{(j)}}(x) = n 2x (1-x^2)^{n-1} } && 0<x<1
\end{align*}
\subsection*{Part B}
With $n=3$, we have three Beta distributed random variables. I think its important to rephrase this question in easier to understand language. We have three random numbers with a beta distribution. What is the probability that the smallest of these numbers is greater than the median of the beta distribution? The Median of the beta distribution is $\tfrac{1}{\sqrt{2}}$.

First we can state our density with $n=3$
\begin{align*}
f_{X_{(j)}}(x) = (3) 2x (1-x^2)^{3-1} = 6x (1-x^2)^{2} && 0<x<1
\end{align*}
With this in mind, we need to solve for how much area under the curve from the median $\tfrac{1}{\sqrt{2}}$ to the end of the domain $1$. As such we get:
\begin{align*}
P(x_{(x_1)} > \tfrac{1}{\sqrt{2}}) = \int_{\tfrac{1}{\sqrt{2}}}^{1} 6x (1-x^2)^{2} dx = \frac{1}{8}
\end{align*}
\subsection*{Part C}


\end{document}

%This is my super simple Real Analysis Homework template

\documentclass{article}
\usepackage[utf8]{inputenc}
\usepackage[english]{babel}
\usepackage{amssymb,amsmath,amsthm}
\usepackage{geometry}
\geometry{legalpaper, margin=1.5in}

\title{STAT-514 Homework 2}
\author{John Hiles}
\date\today
%This information doesn't actually show up on your document unless you use the maketitle command below

\begin{document}
\maketitle %This command prints the title based on information entered above

%Section and subsection automatically number unless you put the asterisk next to them.
\section*{Problem 1}
Let $X_1,...,X_5$ be a random sample of size 5 from a normal population with mean $0 $and variance $1$ and let $\bar{X}=(1/5)\sum_{i=1}^{5}X_i^2$. Let $X_6$ be another independent observation from the same population. Find the distributions of
\begin{enumerate}
\item[A.] $W=\sum_{i=1}^5 X_i^2$
\item[B.] $U=\sum_{i=1}^5 (X_i-X_{\bar{X}})^2$
\item[C.] $T_1 = \sqrt{5} \frac{X_6}{\sqrt{W}}$
\item[D.] $T_2 = 2(5\bar{X}^2)$

\end{enumerate}
\subsection*{Part A}
$W$ simply represents the non averaged sum of the squares of the sample random variables.
\begin{align*}
W=\sum_{i=1}^5 X_i^2 = X_1^2+X_2^2+X_3^2+X_4^2+X_5^2
\end{align*}lt.
Here lets note that these are standard normal distributions. As such, the sum of their squares (i.e. $W$) will be a chi squared random variable with $5$ degrees of freedom.
\begin{align*}
W = \mathcal{X}_5^2
\end{align*}
\subsection*{Part B}
$U$ here is simply the sample Variance except missing the scaling $\frac{1}{n-1}$ factor.
\begin{align*}
S_5 = \frac{1}{4} U
\end{align*}
Logically, inside the parenthesis we have a sum of a standard normal and a negative nonstandard normal. The resulting distribution will be some sort of normal. That distribution will then be squared. If that resulting distribution is normalized the result will be chi squared. As such, our objective should be to find $\mathbb{E}(X_i-\bar{})$

\clearpage

\section*{Problem 2}
 Show that the median of an F-distribution with $(\nu, \nu)$ degrees of freedom is 1. Also,
show that $Q_1Q_3 = 1$, where Q1 and Q3 are the first and third quartiles, respectively, of the $F(\nu,\nu)$
distribution.

Hint: You will have to use the fact that the reciprocal of an F random variable is also an F random
variable.

\subsection*{Part A}

    
\clearpage
\section*{Problem 3}
Suppose $X_1,X_2,...,X_n$ be a random sample from an exponential distribution with mean $\theta$.
\begin{enumerate}
\item[A)] Derive the density for the smallest order statistic $X_{(1)}$. Identify the distribution.
\item[B)] Show that $X_{(1)}$ is independent of $X_{(n)}-X_{(1)}$.
\end{enumerate}
\subsection*{Part A}
   
\clearpage
\section*{Problem 4}
 Let $X_1,...,X_n$ be a random sample of size $n$ from a population with pdf
 \begin{align*}
 f(x) = \begin{cases} 
      \frac{1}{\theta} & 0<x\leq \theta \\
       0 & \text{ otherwise } 
   \end{cases}
\end{align*}
Let $X_{(1)}<...< X_{(n)}$ be the order statistics. Show that $\frac{X(1)}{X(n)}$ and $X_{(n)}$ are independent
random variables.

\subsection*{Part A}



\clearpage
\section*{Problem 5}
Suppose $X_1,...,X_n$ be a random sample from a $Beta(2,1)$ distribution.
\begin{enumerate}
\item[A.] Derive the density for the smallest order statistic $X_{(1)}$
\item[B.] Suppose $n = 3$. Compute the probability that $X_{(1)}$ exceeds the median of the distribution.
\item[C.] Again, let $n = 3$. What is the covariance between $X_{(2)}$ and $X_{(3)}$.

\end{enumerate}

\subsection*{Part A}
All $X_{(n)}$ samples are sampled from the $Beta(2,1)$ distribution. The pdf of this distribution is given by:
\begin{align*}
f(x) = \frac{x^{\alpha-1}(1-x)^{\beta-1}}{B(\alpha,\beta)} = 2x & 0<x<1
\end{align*}




\subsection*{Part B}
Since $n=3$, the only value which exceeds the median is $X_{(3)}$ itself.



\end{document}
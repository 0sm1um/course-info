%This is my super simple Real Analysis Homework template

\documentclass{article}
\usepackage[utf8]{inputenc}
\usepackage[english]{babel}
\usepackage{amssymb,amsmath,amsthm}
\usepackage{geometry}
\geometry{legalpaper, margin=1.5in}

\title{STAT-514 Homework 3}
\author{John Hiles}
\date\today
%This information doesn't actually show up on your document unless you use the maketitle command below

\begin{document}
\maketitle %This command prints the title based on information entered above

%Section and subsection automatically number unless you put the asterisk next to them.
\section*{Problem 1}
Let $X_i$, $i = 1,...,n$ be independent $Bernoulli(p)$ random variables and let $Y_n = \frac{1}{n}\sum_{i=1}^{n} Xi$
\begin{enumerate}
\item[a.] Show that $\sqrt{n}(Y_n-p)\rightarrow^d\mathcal{N}(0,p(1 - p))$
\item[b.] Show that for $p \neq \frac{1}{2}$, the estimate of the variance $Y_n(1-Y_n)$ satisfies
\begin{align*}
\sqrt{n}[Y_n(1 - Yn) - p(1 - p)] \rightarrow^d \mathcal{N}(0,[1 - 2p]^2 p[1 - p])
\end{align*}
\item[c.] Show that for $p=\frac{1}{2}$
\begin{align*}
n[Y_n(1 - Yn)] \rightarrow^d -\frac{1}{4}\mathcal{X}^2_1
\end{align*}
\item[d.] Show that the sample variance can be obtained as
\begin{align*}
S^2 = \frac{n}{n-1} Y_n(1-Y_n)
\end{align*}
\item[e.] Use the results from $(b.)$ and $(d.)$, along with Slutsky’s Theorem, to derive the limiting distribution of an appropriately standardized $S^2$, when $p \neq \frac{1}{2}$.
\item[f.] Find the (approximate?) probability that a random sample of size $n = 100$ from a $Bernoulli(p = 1/4)$ distribution has sample variance greater than $0.2$.
\end{enumerate}
\subsection*{Part A}
%Apply CLT
Note here that $Y_n$ is simply the sample mean $\bar{X}$. $E(X)=p$ since X is a Bernoulli random variable. As such, Central Limit Theorem applies here. Central Limit Theorem states:
\begin{align*}
\lim_{n\rightarrow \infty} \sqrt{n}\frac{\bar{X}_n-\mu}{\sigma} \rightarrow^d \mathcal{N}(0,1)
\end{align*}
Note here that the Variance of $Var(X_i)=p(1-p)$. As such:
\begin{align*}
\boxed{ \sqrt{n}(Y_n-p) \rightarrow^d \mathcal{N}(0,p(1 - p)) }
\end{align*}
\subsection*{Part B}
%Apply delta method
The Delta method states that if the following converges in distribution:
\begin{align*}
\sqrt{n}[Y_n - \theta] \rightarrow^d \mathcal{N}(0,\sigma^2)
\end{align*}
Then for any continuous function $g(x)$ the following is also true:
\begin{align*}
\sqrt{n}[g(Y_n) - g(\theta)] \rightarrow^d \mathcal{N}(0,\sigma^2 [g'(\theta)]^2)
\end{align*}
Let $g(x)=x(1-x)$, $g'(x)=1-2x$ and $\sigma^2=p(1 - p)$ We can apply this to our result in part A:
\begin{align*}
\boxed{ \sqrt{n}[Y_n(1 - Yn) - p(1 - p)] \rightarrow^d \mathcal{N}(0,[1 - 2p]^2 p[1 - p]) }
\end{align*}
Note, that the reason this result is invalid for $p=\frac{1}{2}$ is because at that value, $g'(p)=0$ for which the delta method is not valid for.
\subsection*{Part C}
%Second order delta method
Here we can apply the second order delta method. The second order delta method is given by:
\begin{align*}
n[g(X_n) - g(\theta)] \rightarrow^d \sigma^2 \frac{g''(\theta)}{2} \mathcal{X}^2_{1}
\end{align*}
Let $g(x)=x(1-x)$, $g''(x)=-2$. Note that $g(p) = p(1-p)$, and if $p=1/2$ then  $g(p) = 0$. So our delta method expression is given by:
\begin{align*}
\boxed{ n(Y_n(1-Y_n) - 0) \rightarrow^d -\frac{1}{4} \mathcal{X}^2_{1} }
\end{align*}
\subsection*{Part D}
%Manipulation of sample variance
Sample variance is given by:
\begin{align*}
S^2 = \frac{\sum_{i=1}^{n} (X_i-\bar{X})^2}{n-1}
\end{align*}
$\bar{X}$ is a repeated term in all of these summations, so it can be separated:
\begin{align*}
S^2 = [\frac{1}{n-1} \sum_{i=1}^{n} X_i^2] - \frac{n}{n-1} \bar{X}^2_n = \frac{n}{n-1} [\frac{1}{n} \sum_{i=1}^{n} X_i^2-\bar{X}_n^2]
\end{align*}
Note here we are dealing with Bernoulli random variables, which can only be 1 or 0. As such, $X_i^2=X_i$.
\begin{align*}
S^2 = \frac{n}{n-1} [\frac{1}{n} \sum_{i=1}^{n} X_i-\bar{X}_n^2]
\end{align*}
Recall here that $\bar{X}=\frac{1}{n} \sum_{i=1}^{n} X_i^2 = Y_n$:
\begin{align*}
S^2 = \frac{n}{n-1} [Y_n-Y_n^2]
\end{align*}
Here factoring out a factor of $Y_n$ yields our given result:
\begin{align*}
\boxed{ S^2 = \frac{n}{n-1} Y_n[1-Y_n] }
\end{align*}

\subsection*{Part E}
%
What is this question asking? From what I understand Slutsky's theorem requires one expression which converges in probability and another which converges in distribution. What does appropriately standardized mean?



\subsection*{Part F}
%
I am also unsure how to determine this. I suspect it has to do with some application of CLT, and perhaps some analytically property of the standard normal but beyond that I have no idea. That being said, I can give the expression for the sample variance, and I can say that the true variance should be pretty close to that given that it is a sample of 100 points.


The sample variance predicted by the result in d equals:
\begin{align*}
\boxed{ S^2 = \frac{n}{n-1} Y_n[1-Y_n] = \frac{100}{99} \tfrac{1}{4}[\frac{3}{4}] = 0.1894 }
\end{align*}
\clearpage

\section*{Problem 2}
Suppose $X$ follows a gamma distribution with parameters $\alpha = n$ and $\beta$.
\begin{enumerate}
\item[a.] Show that
\begin{align*}
\lim_{n\rightarrow \infty} P(\frac{X-n\beta}{\sqrt{n\beta^2}} \leq t) = \int_{-\infty}^{t} \frac{1}{2\pi} e^{-\frac{u^2}{2}} du
\end{align*}
\item[b.] Show that $\sqrt{2X} - \sqrt{2n\beta} \rightarrow^d \mathcal{N}(0,\tau)$ as $n\rightarrow \infty$ and specify $\tau$.
\end{enumerate}
\subsection*{Part A}
%Define X = sum y_i where Y is a exponential distribution with mean \beta var \beta^2
%Via delta method/CLT show sample mean converges to N(0,beta^2)
%Set up CLT to equal function h(x)
%Solve for the h(x) which satisfies CLT
%Equate h(x) to the function thats given
%Show that as n\rightarrow infinity H(x) converges to the given integral.
First lets take a look at this given expression.
\begin{align*}
\lim_{n\rightarrow \infty} P(\frac{X-n\beta}{\sqrt{n\beta^2}} \leq t) = \int_{-\infty}^{t} \frac{1}{2\pi} e^{-\frac{u^2}{2}} du
\end{align*}
This is essentially just the statement of the Central Limit theorem since $P(u \leq t)$ is equivalent to the CDF of $\frac{\sqrt{n}(\bar{X}-\mu)}{\sigma}$ at point $u$. Note our random variable is a gamma distribution which has mean $\mu=n\beta$ and Variance $\sigma^2=n^2 \beta^2$. So long as that CDF exists, this expression must be true via the Central limit theorem.
\begin{align*}
\boxed{ \lim_{n\rightarrow \infty} P(\frac{X-n\beta}{\sqrt{n\beta^2}} \leq t) = \int_{-\infty}^{t} \frac{1}{2\pi} e^{-\frac{u^2}{2}} du }
\end{align*}

\subsection*{Part B}
%Via part A we've shown that \sqrt{n}[1/nX-\beta] \rightarrow N(0,\beta^2) as n approaches infinity
%Just apply delta method
In part A we've shown that $\frac{X-n\beta}{\sqrt{n\beta^2}}$ converges in distribution to the standard normal. The denominator of this expression basically serves to normalize the numerator which is $X-n\beta$. The Delta Method states that if the following converges:
\begin{align*}
\sqrt{n}(X-n\beta) \rightarrow^d \mathcal{N}(0,\sigma^2)
\end{align*}
Then the following also converges in distribution:
\begin{align*}
\sqrt{n}(g(X)-g(n\beta)) \rightarrow^d \mathcal{N}(0,\sigma^2 g''(n\beta)^2)
\end{align*}


Let $g(x)=\sqrt{2x}$, via the delta method we can say:
\begin{align*}
\boxed{ \sqrt{n}[\frac{1}{n}\sqrt{2X}-\sqrt{2\beta}] \rightarrow^d \mathcal{N}(0,\beta^2) }
\end{align*}
Here $\tau=\beta^2$

\clearpage

\section*{Problem 3}
 Suppose $X_1,..., X_n$ be a random sample from a gamma distribution with parameters $\alpha = 3$ and $\beta = 1$. Let
 \begin{align*}
 T_n = \frac{n}{\sum_{i=1}^{n}\tfrac{1}{X_i}}
 \end{align*}
 be the harmonic mean of the sample.
 \begin{enumerate}
 \item[a.] Find the constant $\theta$ such that $T_n \rightarrow^P \theta$ as $n\rightarrow \infty$
 \item[b.] Find constants $a$ and $b$ such that
 \begin{align*}
 \frac{\sqrt{n}(T_n-a)}{b} \rightarrow^d \mathcal{N}(0,1)
 \end{align*}
 as $n\rightarrow \infty$.
 \end{enumerate}
\subsection*{Part A}
%T_n is not bar(X_n). It is the mean of mean(1/X_i). Harmonic mean converges in probability to E(1/X) due to weak law of large numbers. Result should be 1/2
%From WLLN Bar{Xn} converges to mu as x approaches infinity.
%But Tn converges in probability to E(X^-1) as n approaches infinity, therefore theta = E(1/X)
This first question asks to find what value the harmonic mean converges to in probability. This value will range between 0 and 1. Our sample is one from a Gamma(3,1) random variable. Via the WLLN, $\bar{X} \rightarrow^p \mathbb{E}(X)$, the Harmonic mean then converges to $\mathbb{E}(\frac{1}{X})$.
\begin{align*}
\mathbb{E}(\tfrac{1}{X}) = \int_{0}^{\infty} \frac{1}{x} f(x) dx
\end{align*}
Note here that the pdf of Gamma(3,1) is given by $f(x)=\tfrac{1}{2}e^{-x}x^2$. So our integral is:
\begin{align*}
\mathbb{E}(\tfrac{1}{X}) = \tfrac{1}{2} \int_{0}^{\infty} \frac{1}{x} e^{-x}x^2 dx = \frac{1}{2}
\end{align*}
As such:
\begin{align*}
\boxed{ T_n \rightarrow^P \frac{1}{2} }
\end{align*}
Where $\theta=\frac{1}{2}$
\subsection*{Part B}
This here is a central limit theorem problem. Note that convergence in probability implies convergence in distribution(but not the other way around.).
\begin{align*}
\frac{\sqrt{n}(T_n-a)}{b} \rightarrow^d \mathcal{N}(0,1)
\end{align*}
For this problem, $T_n$ can be treated as the mean of a new random variable $\frac{1}{X}$. The mean of $\frac{1}{X}=\mathbb{E}(\frac{1}{X})=\frac{1}{2}$. However we still need $\sigma=\sqrt{Var(\frac{1}{X})}$. If we treat $\frac{1}{X}$ we can apply CLT to it to arrive at a standard normal. We just need the Variance.
\begin{align*}
Var(\frac{1}{X}) = \mathbb{E}(\frac{1}{X^2}) - \mathbb{E}(X)^2 = \tfrac{1}{2} \int_{0}^{\infty} \frac{1}{x^2} e^{-x}x^2 dx - \frac{1}{2^2} = \frac{1}{2} - \frac{1}{4} = \frac{1}{4}
\end{align*}
As such, $a=\frac{1}{2}$, $b=\frac{1}{2}$
\begin{align*}
\boxed{ 2\sqrt{n}(T_n-\frac{1}{2}) \rightarrow^d \mathcal{N}(0,1) }
\end{align*}
\clearpage

\section*{Problem 4}
Let $X_1,...,X_n$ be independent random variables with densities:
\begin{align*}
f(x|\theta,\sigma) = \begin{cases} 
      e^{i\theta - x} & x \geq i\theta \\
       0 & x < i\theta 
   \end{cases}
\end{align*}
Prove that $T = min_i(\frac{X_i}{i})$ is a sufficient statistic for $\theta$.
\subsection*{Part A}
It is important to note here that $X_i,...X_n$ is not an iid sample of random variables but a sequence of random variables. That being said, they are still independent. And as such, their joint distribution must then be the product of their individual pdfs.
\begin{align*}
f_{X_1,...,X_n}(X_1,...,X_n) = (e^{\theta - x}) (e^{2\theta - x}) ... (e^{i\theta - x}) = \prod_{i=1}^{n} e^{i\theta - x_i} \mathbf{1}\{x_{(1)} \geq \theta\}
\end{align*}
Note here that $\mathbf{1}$ denotes the indicator function which equals 1 when the condition is met and zero elsewhere. Another important thing to note here is that the function essentially says that in order to exist, the first order statistic (smallest) must be greater than $i\theta$. This essentially restricts the domain to conform with the given pdf. Furthermore, the product can be rearranged as a summation due to properties of exponents:
\begin{align*}
f_{X_1,...,X_n}(X_1,...,X_n) = e^{\sum_{i=1}^{n} i\theta - x_i} \prod_{i=1}^{n} \mathbf{1}\{x_{(1)} \geq i\theta\} = e^{\sum_{i=1}^{n}i\theta} e^{-\sum_{i=1}^{n} x_i} \prod_{i=1}^{n} \mathbf{1}\{x_{(1)} \geq \theta\}
\end{align*}
At this point let me draw your attention to the first order statistic $x_{(1)}$. The statistic given in the problem is $T = min_i(\frac{X_i}{i})$. Note that the statistic T is equivilant to the first order statistic of the sequence of first order statistic. $T=X_{(1)}$. As such we can substitute the two:
\begin{align*}
f_{X_1,...,X_n}(X_1,...,X_n) = e^{\sum_{i=1}^{n}i\theta} e^{-\sum_{i=1}^{n} x_i} \prod_{i=1}^{n} \mathbf{1}\{min_i(\frac{X_i}{i}) \geq i\theta\} = e^{\sum_{i=1}^{n}i\theta} e^{-\sum_{i=1}^{n} x_i} \prod_{i=1}^{n} \mathbf{1}\{min_i(\frac{X_i}{i}) \geq \theta\}
\end{align*}
In our current form, this is factorized according to Fischer-Neyman Factorization Theorem.
\begin{align*}
f(x|\theta) = g(T(x)|\theta)h(x)
\end{align*}
Where $g(T(x)|\theta) = e^{\sum_{i=1}^{n}i\theta} \prod_{i=1}^{n} \mathbf{1}\{min_i(\frac{X_i}{i}) \geq \theta\}$. and $h(x)=e^{-\sum_{i=1}^{n} x_i}$. Therefore
\begin{align*}
\boxed{ \text{T is a sufficient statistic of } \theta }
\end{align*}


\clearpage

\section*{Problem 5}
Let $X_1,...,X_n$ be a random sample from the pdf
\begin{align*}
f(x|\theta,\sigma) = \begin{cases} 
      \frac{1}{\sigma} e^{-\frac{(x-\theta)}{\sigma}} & x > \theta \\
       0 & x < \theta 
   \end{cases}
\end{align*}
where $0<\sigma<\infty$. Find a two dimensional sufficient statistic for $(\theta,\sigma^2)$.
\subsection*{Part A}
Fischer-Neyman Factorization Theorem states that a statistic $T(X)$ is a sufficient statistic if it can be factored in the following format:
\begin{align*}
f(x|\theta) = g(T(x)|\theta)h(x)
\end{align*}
Where $f(x|\theta)$ is the joint pdf of the sample, and $h(x)$ is any arbitrary function. Given that our sample is iid, its joint distribution is the product of each of the members of the sample together:
\begin{align*}
f_{X_1,...,X_i}(x|\theta,\sigma) = \prod_{i=1}^{n} [\frac{1}{\sigma} e^{-\frac{(x_i-\theta)}{\sigma}}] \mathbf{1}\{x_i > \theta\}
\end{align*}
Note that $\mathbf{1}\{x > \theta\}$ denotes the indicator function which equals 1 when the condition is met and zero elsewhere.
\begin{align*}
f_{X_1,...,X_i}(x|\theta,\sigma) = \frac{1}{\sigma^n} e^{\frac{n\theta}{\sigma}} e^{-\frac{\sum_{i=1}^{n} x_i}{\sigma}} \prod_{i=1}^{n} \mathbf{1}\{x_i > \theta\} \mathbf{1}\{x_i > \theta\}
\end{align*}
Lets take a look at our indicator function order statistics for a second. Just like in problem 4, we know that we can express our indicator function in terms of order statistics. And we can equivalently represent the bounds of the pdf like this:
\begin{align*}
f_{X_1,...,X_i}(x|\theta,\sigma) = \frac{1}{\sigma^n} e^{\frac{n\theta}{\sigma}} e^{-\sum_{i=1}^{n} x_i} \mathbf{1}\{x_{(1)} > \theta\} \prod_{i=1}^{n} \mathbf{1}\{x_i > \theta\}
\end{align*}
This pdf is now able to be factored. Let $T_1 = \sum_{i=1}^{n} x_i$ and $T_2 = x_{(1)}$ Let $T = (T_1,T_2)$ Then:
\begin{align*}
g(T_1,T_2|\theta,\sigma) = \frac{1}{\sigma^n} e^{\frac{n\theta}{\sigma}} e^{-\sum_{i=1}^{n} x_i} \mathbf{1}\{x_{(1)} > \theta\} && h(x)=\prod_{i=1}^{n} \mathbf{1}\{x_i > \theta\}
\end{align*}
\begin{align*}
\boxed{ \text{T is a sufficient statistic of } \theta,\sigma }
\end{align*}

\clearpage

\section*{Problem 6}
For each of the following distributions let $X_1,...,X_n$ be a random sample. Find a minimal sufficient statistic for $\theta$.
\begin{enumerate}
\item[a.] $f(x|\theta) = \frac{1}{\sqrt{2\pi}} e^{-(x-\theta)^2}$, $-\infty < x < \infty$, $-\infty < \theta < \infty$
\item[b.] $f(x|\theta) = e^{-(x-\theta)}$, $\theta < x < \infty$, $-\infty < \theta < \infty$
\item[c.] $f(x|\theta) = \frac{e^{-(x-\theta)}}{1+e^{-(x-\theta)^2}} $, $-\infty < x < \infty$, $-\infty < \theta < \infty$
\end{enumerate}
\subsection*{Part A}
Lets start by finding the joint pdf of this expression. The joint pdf of these iid random variables is their product:
\begin{align*}
f(x|\theta) = \prod_{i=1}^{n} \frac{1}{\sqrt{2\pi}} e^{-(x_i-\theta)^2} = \frac{1}{(\sqrt{2\pi})^n} e^{-\sum_{i=1}^{n}(x_i-\theta)^2}
\end{align*}
Expanding out the square in the exponent:
\begin{align*}
f(x|\theta) = \frac{1}{(\sqrt{2\pi})^n} e^{-\sum_{i=1}^{n} [x_i^2 - 2x_i\theta + \theta^2]}
\end{align*}
Separating our sum into three exponential functions:
\begin{align*}
f(x|\theta) = \frac{1}{(\sqrt{2\pi})^n}  e^{n\theta^2} e^{-\sum_{i=1}^{n} x_i^2} e^{\sum_{i=1}^{n} 2x_i\theta}
\end{align*}
Pulling out factors of theta:
\begin{align*}
f(x|\theta) = \frac{1}{(\sqrt{2\pi})^n}  e^{n\theta^2} e^{n\theta}  e^{-\sum_{i=1}^{n} x_i^2} e^{2\sum_{i=1}^{n} x_i}
\end{align*}
Fischer-Neyman Factorization Theorem states that a statistic $T(X)$ is a sufficient statistic if it can be factored in the following format:
\begin{align*}
f(x|\theta) = g(T(x)|\theta)h(x)
\end{align*}
Let $T(x)=\sum_{i=1}^{n} x_i^2$ $g(T(x)|\theta)=\frac{1}{(\sqrt{2\pi})^n}  e^{n\theta^2} e^{n\theta}  e^{-\sum_{i=1}^{n} x_i^2}$ and $h(x) = e^{2\sum_{i=1}^{n} x_i}$. To test minimal sufficiency we need to test whether or not the ratio of $\frac{g(T(x)|\theta)}{g(T(y)|\theta)}$ depends on $\theta$.
\begin{align*}
\frac{f(x|\theta)}{f(y|\theta)} = \frac{\frac{1}{(\sqrt{2\pi})^n}  e^{n\theta^2} e^{n\theta}  e^{-\sum_{i=1}^{n} x_i^2}}{\frac{1}{(\sqrt{2\pi})^n}  e^{n\theta^2} e^{n\theta}  e^{-\sum_{i=1}^{n} y_i^2}} = \frac{e^{-\sum_{i=1}^{n} x_i^2}}{e^{-\sum_{i=1}^{n} y_i^2}}
\end{align*}
Hence
\begin{align*}
\boxed{ \text{T is a minimal sufficient statistic of } \theta}
\end{align*}

\subsection*{Part B}
Lets start by finding the joint pdf of this expression. The joint pdf of these iid random variables is their product:
\begin{align*}
f(x|\theta) = \prod_{i=1}^{n} e^{-(x_i-\theta)} \mathbf{1}\{\theta < x\} = e^{-\sum_{i=1}^{n} (x_i-\theta)} \mathbf{1}\{\theta < x_i\}
\end{align*}
Pulling out $\theta$ from the summation:
\begin{align*}
f(x|\theta) = e^{n\theta} e^{-\sum_{i=1}^{n} (x_i)} \mathbf{1}\{\theta < x_i\}
\end{align*}
Fischer-Neyman Factorization Theorem states that a statistic $T(X)$ is a sufficient statistic if it can be factored in the following format:
\begin{align*}
f(x|\theta) = g(T(x)|\theta)h(x)
\end{align*}
Let $T(x)=\sum_{i=1}^{n} x_i$ $g(T(x)|\theta)= e^{n\theta} e^{-\sum_{i=1}^{n} (x_i)} \mathbf{1}\{\theta < x_i\}$ and $h(x) = 1$. To test minimal sufficiency we need to test whether or not the ratio of $\frac{g(T(x)|\theta)}{g(T(y)|\theta)}$ depends on $\theta$.
\begin{align*}
\frac{f(x|\theta)}{f(y|\theta)} = \frac{e^{n\theta} e^{-\sum_{i=1}^{n} (x_i)} \mathbf{1}\{\theta < x_i\}}{e^{n\theta} e^{-\sum_{i=1}^{n} (y_i)} \mathbf{1}\{\theta < y_i\}}
\end{align*}
One way to remove the dependence on $\theta$ in this expression is to re frame this indicator function in terms of the first order statistic. By definition the first order statistic must be greater than $\theta$.
\begin{align*}
\boxed{ \frac{f(x|\theta)}{f(y|\theta)} = \frac{e^{-\sum_{i=1}^{n} (x_i)} \mathbf{1}\{x_{(1)} < x_i\}}{e^{n\theta} e^{-\sum_{i=1}^{n} (y_i)} \mathbf{1}\{x_{(1)} < y_i\}} }
\end{align*}
In this case, $T(x)=x_{(1)}$ would then be the minimal sufficient first order statistic.

\subsection*{Part C}
Lets start by finding the joint pdf of this expression. The joint pdf of these iid random variables is their product:
\begin{align*}
f(x|\theta) = \prod_{i=1}^{n} \frac{ e^{-(x_i-\theta)} }{1+e^{-(x_i-\theta)^2}} = \frac{ e^{-\sum_{i=1}^{n}(x_i-\theta)} }{1+e^{-\sum_{i=1}^{n}(x_i-\theta)^2}}
\end{align*}

\begin{align*}
f(x|\theta) = \frac{e^{n\theta} e^{-\sum_{i=1}^{n}(x_i)} }{1+e^{-\sum_{i=1}^{n} (x_i^2-2\theta x_i+\theta^2)}}
\end{align*}

Fischer-Neyman Factorization Theorem states that a statistic $T(X)$ is a sufficient statistic if it can be factored in the following format:
\begin{align*}
f(x|\theta) = g(T(x)|\theta)h(x)
\end{align*}
Let $T(x)=\sum_{i=1}^{n} (x_i^2-2\theta x_i)$ $g(T(x)|\theta)= \frac{e^{n\theta}}{1+e^{{-\sum_{i=1}^{n} (x_i^2-2\theta x_i+\theta^2)}}} $ and $h(x) = e^{-\sum_{i=1}^{n}(x_i)}$. To test minimal sufficiency we need to test whether or not the ratio of $\frac{g(T(x)|\theta)}{g(T(y)|\theta)}$ depends on $\theta$.
\begin{align*}
\frac{f(x|\theta)}{f(y|\theta)} = \frac{\frac{e^{n\theta}}{1+e^{{-\sum_{i=1}^{n} (x_i^2-2\theta x_i+\theta^2)}}}}{\frac{e^{n\theta}}{1+e^{{-\sum_{i=1}^{n} (y_i^2-2\theta y_i+\theta^2)}}}} = \frac{1+e^{{-\sum_{i=1}^{n} (x_i^2-x_i)}}}{1+e^{{-\sum_{i=1}^{n} (y_i^2-y_i)}}}
\end{align*}

Hence
\begin{align*}
\boxed{ \text{T is a minimal sufficient statistic of } \theta}
\end{align*}

\clearpage
\end{document}

%This is my super simple Real Analysis Homework template

\documentclass{article}
\usepackage[utf8]{inputenc}
\usepackage[english]{babel}
\usepackage{amssymb,amsmath,amsthm}
\usepackage{graphicx}
\graphicspath{ {figures/} }
\usepackage{geometry}
\geometry{legalpaper, margin=1.5in}

\title{STAT-636 Homework 4}
\author{John Hiles}
\date\today
%This information doesn't actually show up on your document unless you use the maketitle command below

\begin{document}
\maketitle %This command prints the title based on information entered above

%Section and subsection automatically number unless you put the asterisk next to them.
\section*{Problem 1}
\subsection*{A}
Done!
\section*{Problem 2}
For the following constrained optimization problem:
\begin{align*}
min [(x_1 - 2)^2 + (x_2 - 1)^2] && \text{ subject to } -x_1 + x_2 = 3
\end{align*}
\begin{enumerate}
\item[a.] Write the Lagrange Function of this problem.
\item[b.] Using KKT conditions, write the 3 equations that a global minimum should satisfy.
\end{enumerate}
\subsection*{A}
The Lagrange expression is given by:
\begin{align*}
\mathcal{L}(x_1,x_2,\lambda) = f(x_1,x_2) - \lambda g(x_1,x_2)
\end{align*}
Where $f(x_1,x_2)$ is the objective function and $g(x_1,x_2)$ is the equality constraint arranged to equal zero. Substituting in our given objective function and constraint yields:
\begin{align*}
\boxed{ \mathcal{L}(x_1,x_2,\lambda) = (x_1 - 2)^2 + (x_2 - 1)^2 - \lambda (-x_1 + x_2 - 3) }
\end{align*}
\subsection*{B}
Our equations involve differentiating this expression with respect to each parameters and setting each equal to zero.
\begin{align*}
0 = \frac{\partial}{\partial x_1}[(x_1 - 2)^2 + (x_2 - 1)^2 - \lambda (-x_1 + x_2 - 3)] && \\
0 = \frac{\partial}{\partial x_2}[(x_1 - 2)^2 + (x_2 - 1)^2 - \lambda (-x_1 + x_2 - 3)] && \\
0 = \frac{\partial}{\partial \lambda}[(x_1 - 2)^2 + (x_2 - 1)^2 - \lambda (-x_1 + x_2 - 3)]
\end{align*}
Evaluating these derivatives yields the following equations:
\begin{align*}
\boxed{0 = 2x_1+\lambda-4} && \\
\boxed{0 = 2x_2+\lambda-2} && \\
\boxed{0 = x_1+y_1+3}
\end{align*}
\clearpage

\section*{Problem 3}
My particular PCA summary is given by:
\begin{figure}[hbt!]
\centering
\includegraphics[width=0.7\linewidth]{PCASummary.png}
\caption{PCA Summary given in R}
\end{figure}
\subsection*{Part A}
The Barplot of the relative variances is here:
\begin{figure}[hbt!]
\centering
\includegraphics[width=0.7\linewidth]{barplot.png}
\caption{Bar plot of scaled variances given in R}
\end{figure}
\subsection*{Part B}
The cumulative proportion of variance of Principal components 1 through 9 is 87.46 percent of variance. Thus components 1 through 9 should be included to account for at least 85\% of the variance.
\subsection*{Part C}
41.56 percent of the total variance are explained by the first two principal components.
\subsection*{Part D}
The scatterplot of the first two principal components is given by:
\begin{figure}[hbt!]
\centering
\includegraphics[width=0.7\linewidth]{scatterplot.png}
\caption{First two principal components given in R}
\end{figure}
\end{document}

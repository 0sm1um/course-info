%This is my super simple Real Analysis Homework template

\documentclass{article}
\usepackage[utf8]{inputenc}
\usepackage[english]{babel}
\usepackage{amssymb,amsmath,amsthm}
\usepackage{graphicx}
\usepackage{tabularx}
\graphicspath{ {figures/} }
\usepackage{geometry}
\geometry{legalpaper, margin=1.5in}

\title{STAT-636 HW 07}
\author{John Hiles}
\date\today
%This information doesn't actually show up on your document unless you use the maketitle command below

\begin{document}
\maketitle %This command prints the title based on information entered above


\section*{Problem 1}
\textbf{Ch 7}
Done!

\section*{Problem 2}
In this problem, you will perform k-means clustering manually, with K = 2, on a small example
with n = 10 observations and p = 1 variables. The observations are as follows.
  \begin{tabular}{c|c|c|c|c|c|c|c|c|c|c}
    \hline
    \hline
    Obs. & 1 & 2 & 3 & 4 & 5 & 6 & 7 & 8 & 9 & 10\\
    \hline
    x & 0 & 1 & 2 & 3 & 6 & 7 & 7 & 8 & 10 & 10\\
    \hline
    \hline
  \end{tabular}
\begin{enumerate}
\item[a.] Assume that the initial center points are $x_1 = 5$ and $x_2 = 10$. Classify the 10 observations into 2 classes according to the Euclidean distance.
\item[b.] Update the center points $x_1$ and $x_2$ and reclassify the observations
\end{enumerate}
\subsection*{Part A.}
The K-Means algorithm can be outlined in psuedocode as:
\begin{enumerate}
\item[•] Select K points as initial centroids.
\item[•] Loop:
\item[•] Form K Clusters by assigning each point to the closest centroid.
\item[•] Recompute the centroid of each cluster.
\item[•] Break Loop when Centroids do not change.
\end{enumerate}
The problem gives us $K=2$ centroids, located at $x_1=5$ and $x_2=10$. Our two clusters are:
\begin{align*}
c_1 = \{0,1,2,3,6,7,7 \} && c_2 = \{ 8, 10, 10\}
\end{align*}
Next step is to update $x_1$ and $x_2$ to the new centers of each clusters. The new $x_1=3.71$ and the new centroid of $x_2=9.33$. The new clusters are:
\begin{align*}
\boxed{ c_1 = \{ 0,1,2,3,6 \} } && \boxed{ c_2 = \{ 7, 7, 8, 10, 10 \} }
\end{align*}


\clearpage

\section*{Problem 3}
Suppose that we have four observations, for which we compute a dissimilarity (distance) matrix, given by:
\begin{align*}
\begin{bmatrix}
 0   & 0.5 & 0.6  & 0.9  \\
 0.5 &  0  & 0.7  & 1.0  \\
 0.6 & 0.7 &  0   & 0.65 \\
 0.9 & 1.0 & 0.65 &  0   \\
\end{bmatrix}
\end{align*}
For instance, the dissimilarity (distance) between the first and second observations is $0.5$, and the dissimilarity (distance) between the second and fourth observations is $1.0$.
\begin{enumerate}
\item[a.] Sketch the dendrogram that results from hierarchically clustering these four observations using complete (MAX) linkage.
\item[b.] Repeat (a), this time using single (MIN) linkage clustering.
\item[c.] Suppose that we cut the dendograms obtained in (a) and (b) to get two clusters. What
clustering results will we get in these two cases?
\end{enumerate}
\subsection*{Part A.}
The MAX linkage clustering algorithm is given by.
\begin{enumerate}
\item[1.] Compute, or just have a proximity matrix.
\item[2.] Loop:
\item[•] Merge 2 closest clusters.
\item[•] Re link with other clusters via \textbf{furthest} distance between cluster members.
\item[•] Update proximity matrix to reflect the proximity between the new closter and the original clusters.
\item[3.] Break Loop when one cluster remains.
\end{enumerate}
To manually compute this algorithm, I felt it more intuitive to draw out the 4 measurements in "space". Note that each diagram is one "step" of the loop. So $I_1$ is iteration 1, $I_2$ is the second and so on. Note that I denote each measurement by $z_i$ where i is their respective subscript.

\begin{figure}[hbt!]
\centering
\includegraphics[width=0.7\linewidth]{MeasInSpace.jpg}
\caption{Measurements in Space}
\end{figure}
\begin{figure}[hbt!]
\centering
\includegraphics[width=0.7\linewidth]{partADendogram.jpg}
\caption{Dendogram for Part A.}
\end{figure}
\clearpage

\subsection*{Part B.}
The MIN linkage clustering algorithm is given by.
\begin{enumerate}
\item[1.] Compute, or just have a proximity matrix.
\item[2.] Loop:
\item[•] Merge 2 closest clusters.
\item[•] Re link with other clusters via \textbf{closest} distance between cluster members.
\item[•] Update proximity matrix to reflect the proximity between the new closter and the original clusters.
\item[3.] Break Loop when one cluster remains.
\end{enumerate}
Here is the clustering of this part:

\begin{figure}[hbt!]
\centering
\includegraphics[width=0.7\linewidth]{PartBClustering.jpg}
\caption{The clustering procedure.}
\end{figure}
\begin{figure}[hbt!]
\centering
\includegraphics[width=0.7\linewidth]{partBDendogram.jpg}
\caption{Dendogram for Part B.}
\end{figure}

This scheme has resulted in an entirely different dendogram.

\clearpage
\subsection*{Part C.}
The clusters would essentially be the top two clusters of each dendogram.
So $\{z_1,z_2 \},\{ x_3,x_4 \}$ and $\{z_1,z_2,z_3 \},\{ z_4 \}$


\section*{Problem 4}
Import the criminal rests data of the 50 states using data(USArrests).
\begin{enumerate}
\item[a.] Draw a dendrogram using agnes with the default setting. (Do not need to scale the data.)
\item[b.] Cut the dendrogram to get two clusters. Are Virginia and Maryland in the same group or not? (Hint: Virginia is the 46th observation and Maryland is the 20th.)
\item[c.] Try the k-means clustering with k = 2. (Do not need to scale the data.) In this case, are Virginia and Maryland in the same group or not?
\end{enumerate}

\subsection*{Part A.}
Here is the dendogram. It is not easy to make sense of:
\begin{figure}[hbt!]
\centering
\includegraphics[width=0.7\linewidth]{unscaleddendogram.png}
\caption{Cursed Unscaled Dendogram}
\end{figure}
\subsection*{Part B.}
After cutting, Maryland and Virginia are in different groups.
\begin{figure}[hbt!]
\centering
\includegraphics[width=0.7\linewidth]{partB.png}
\caption{Source Code used for part B.}
\end{figure}

\subsection*{Part C.}
After kmeans clustering, VA and Maryland are still not in the same group.

\begin{figure}[hbt!]
\centering
\includegraphics[width=0.7\linewidth]{partc.png}
\caption{Source Code used for part C.}
\end{figure}

\end{document}
